\documentclass[12pt,article,a4paper,brazil,oldfontcommands,oneside]{abntex2}
\usepackage{lmodern}			% Usa a fonte Latin Modern
\usepackage[T1]{fontenc}		% Selecao de codigos de fonte.
\usepackage[utf8]{inputenc}		% Codificacao do documento (conversão automática dos acentos)
\usepackage{indentfirst}		% Indenta o primeiro parágrafo de cada seção.
\usepackage{nomencl} 			% Lista de simbolos
\usepackage{color}				% Controle das cores
\usepackage{graphicx}			% Inclusão de gráficos
\usepackage{microtype} 			% para melhorias de justificação
\usepackage{rotating}
\usepackage[alf]{abntex2cite}	% Citações padrão ABNT

% Configuração de Margens e espaços
\setlrmarginsandblock{3cm}{3cm}{*}
\setulmarginsandblock{3cm}{3cm}{*}
\setlength{\ABNTEXcitacaorecuo}{1.8cm}
\checkandfixthelayout
\setlength{\parindent}{1.3cm}			% O tamanho do parágrafo é dado por:
\setlength{\parskip}{0.2cm}  			% Controle do espaçamento entre um parágrafo e outro:
%\SingleSpacing

% Informações de dados para CAPA e FOLHA DE ROSTO
\titulo{Análise de opiniões relacionadas a \\notícias em redes sociais}
\autor{Anderson Pontes Batalha \and Orientador: Ivo Marcos Riegel}
\local{Araquari -- SC -- Brasil}
\data{Março de 2018}

% Configurações de aparência do PDF final
\makeatletter
\hypersetup{
     	%pagebackref=true,
		pdftitle={\@title}, 
		pdfauthor={\@author},
    	pdfsubject={Modelo de projeto e TCC},
	    pdfcreator={LaTeX with abnTeX2},
		pdfkeywords={abnt}{latex}{abntex2}{modelo}{TCC}, 
		colorlinks=true,       		% false: boxed links; true: colored links
    	linkcolor=blue,          	% color of internal links
    	citecolor=blue,        		% color of links to bibliography
    	filecolor=magenta,      	% color of file links
		urlcolor=blue,
		bookmarksdepth=4
}
\makeatother

% compila o indice
\makeindex


% Início do documento
\begin{document}

\frenchspacing 		% Retira espaço extra obsoleto entre as frases.
\maketitle			% página de titulo


\section{Tema}

O presente trabalho está voltado a análise de sentimentos de redes sociais. A ideia principal é extrair postagens do Facebook e Twitter relacionados a fatos recentes de grande repercussão e determinar a polaridade das mensagens.

\subsection{Delimitação do Tema}

Durante a realização do trabalho, pretende-se escolher fatos de grande repercussão nas redes sociais, relacionadas a esportes, política, entretenimento e tecnologia, realizando a coleta de dados referentes as mensagens dos usuários que emitiram algum tipo de opiniões sobre tais notícias. Em seguida, utilizar diferentes métodos para análise de sentimentos, e comparar os resultados obtidos por cada método, ressaltando suas vantagens e desvantagens, além de apresentar os resultados. 

Este trabalho irá seguir a abordagem léxica, uma das formas de extrair sentimentos. Através de um dicionário de palavras, onde cada uma possui uma polaridade ou sentimento associado. \cite{taboada2011lexicon} Os métodos utilizados serão o Opinion Lexicon, SentiWordNet, LIWC, SenticNet, PANAS-t e Vader. \cite{benevenuto2015metodos}

\section{Problema}

A grande quantidade de dados gerados diariamente nas redes sociais se constitui como uma boa oportunidade de pesquisa relacionada a análise de sentimentos. As empresas utilizam as mídias sociais como parâmetro para saber qual a reputação da marca perante os seus consumidores, além da receptividade destes a um lançamento de produto ou uma nova campanha publicitária. Também é possível verificar as reações das pessoas a determinados acontecimentos e obter o sentimento expresso por cada mensagem.

Apesar de existirem trabalhos relacionados a esta área, poucos se destinam a analisar sentimentos relacionados a notícias. Por isso, pretende-se estabelecer parâmetros para definir qual método de análise de sentimentos apresenta melhores resultados. 

\section{Justificativa}

Com o advento da Internet, e a popularização dos smartphones, as redes sociais vem ganhando cada vez mais espaço. Isso ocasionou um imenso volume de dados gerados pelas redes sociais. 

Segundo \cite{socialmediatrends}, dados de 2017 dão conta de que 94\% das empresas estão presentes nas redes sociais, a maioria delas acredita que isso exerce um papel importante em seus negócios.

O Facebook tem se consolidado como a maior rede social do mundo, com cerca de 1,94 bilhão de usuários em 2017 \cite{Facebook75:online}. Embora o Twitter tenha apresentado queda no número de usuários, vem se recuperando, e voltou a apresentar aumento, \cite{Twittert93:online} por isso ainda é uma importante mídia social, por permitir a comunicação em tempo real entre as pessoas.

As mídias sociais permitem a interação entre seus usuários, onde é possível trocar mensagens, comunicar-se com outras pessoas com interesses em comum, além de emitir opiniões sobre os mais variados temas, como política, religião, esporte, música, entre outros.

Empresas, governos e pesquisadores começaram a perceber o potencial destes dados, sendo possível analisar qual a reputação de uma marca em relação aos consumidores, ou até mesmo a recepção das pessoas a um determinado lançamento de produto.

A análise de sentimentos tem se tornado uma grande oportunidade de estudo pelos motivos expostos acima.

\section{Objetivos}
\subsection{Objetivo Geral do Trabalho}

Aplicar os métodos de detecção de sentimentos e definir qual obteve melhor desempenho utilizando os dados coletados.

\subsection{Objetivos Específicos}

\begin{itemize}
    \item Selecionar as notícias que serão utilizadas como referência para a coleta de dados.
    \item Pesquisar em artigos relacionados os principais métodos de detecção de sentimentos.
    \item Definir um método para coleta dos dados, e por quanto tempo será realizada.
    \item Realizar a limpeza e pré-processamento dos dados, retirando palavras que não exprimem sentimentos, além de gírias e abreviações.
    \item Definir quais métricas para avaliação dos algoritmos de análise de sentimentos. Existem três métricas para comparação dos resultados: capacidade de predição correta da polaridade de cada mensagem, abrangência e concordância \cite{araujometodos}.
    \item Utilizar aprendizagem de máquina não supervisionada, onde um dicionário léxico possui um conjunto de termos, e cada um dos termos está associado a um sentimento. Este método se apresenta mais eficiente em comparação com os métodos supervisionados. \cite{benevenuto2015metodos}
\end{itemize}

\section{Metodologia}

\begin{enumerate}
    \item Escolha das notícias
    
Será realizada a definição dos acontecimentos que servirão de base para coleta de dados.

    \item Coleta dos dados

Os dados serão coletados do Twitter e do Facebook. Serão utilizados a \href{https://developer.twitter.com/en/docs/api-reference-index}{API do Twitter} (através da biblioteca Python \href{http://www.tweepy.org/}{Tweepy}) e a \href{https://developers.facebook.com/docs/graph-api?locale=pt_BR}{Graph API}, do Facebook. Também serão desenvolvidos scripts para manipular com as APIs.
    
    \item Limpeza dos dados

Ocorrerá a remoção de palavras ou expressões que possam interferir no processo de análise, tais como gírias e abreviações (stop words).

    \item Análise de sentimentos

Aplicação dos métodos de análise de sentimentos no conjunto de dados, e a definição de quais métricas a serem utilizadas para avaliar os resultados de cada método. Pretende-se usar bibliotecas Python como o NLTK (\url{https://www.nltk.org/}) e TextBlob (\url{http://textblob.readthedocs.io/en/dev/} para processamento de linguagem natural.

    \item Resultados

Estabelecer uma comparação, com base nos resultados obtidos, de cada um métodos, apontando suas principais características, pontos fortes e fracos.

\end{enumerate}


\section{Cronograma}

\begin{table}[ht]
\centering
\caption{Cronograma de atividades proposto}
\label{tab:cronograma}
\begin{tabular}{|c|c|c|c|c|c|c|c|}
\hline
\textbf{Atividade}                                                                              & \textbf{Mês 1} & \textbf{Mês 2} & \textbf{Mês 3} & \textbf{Mês 4} & \textbf{Mês 5} & \textbf{Mês 6} & \textbf{Mês 7} \\ \hline
\begin{tabular}[c]{@{}c@{}}Levantamento \\ do tema\end{tabular} & \textbullet & \textbullet & & & & & \\ \hline
\begin{tabular}[c]{@{}c@{}}Pesquisa \\ do tema\end{tabular} 
& \textbullet & \textbullet & & & & & \\ \hline
\begin{tabular}[c]{@{}c@{}}Escolha das\\ notícias\end{tabular}
& & & \textbullet & & & & \\ \hline
\begin{tabular}[c]{@{}c@{}}Coleta dos \\ dados\end{tabular} 
& & & \textbullet & & & & \\ \hline
\begin{tabular}[c]{@{}c@{}}Limpeza dos\\ dados\end{tabular}     & & & & \textbullet & & & \\ \hline
\begin{tabular}[c]{@{}c@{}}Aplicação dos\\ métodos de \\ análise de \\ sentimentos\end{tabular}
& & & & \textbullet & \textbullet & & \\ \hline
Resultados                                                      & & & & & & \textbullet & \\ \hline
Conclusão                                                       & & & & & & \textbullet & \\ \hline
Apresentação                                                    & & & & & & & \textbullet \\ \hline
\end{tabular}
\end{table}


\bibliography{refs}


\end{document}
